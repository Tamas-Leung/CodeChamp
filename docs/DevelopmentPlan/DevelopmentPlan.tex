\documentclass{article}

\usepackage{booktabs}
\usepackage{tabularx}

\title{Development Plan\\CodeChamp}

\author{\authname}

\date{}

%% Comments

\usepackage{color}

\newif\ifcomments\commentstrue %displays comments
%\newif\ifcomments\commentsfalse %so that comments do not display

\ifcomments
\newcommand{\authornote}[3]{\textcolor{#1}{[#3 ---#2]}}
\newcommand{\todo}[1]{\textcolor{red}{[TODO: #1]}}
\else
\newcommand{\authornote}[3]{}
\newcommand{\todo}[1]{}
\fi

\newcommand{\wss}[1]{\authornote{blue}{SS}{#1}} 
\newcommand{\plt}[1]{\authornote{magenta}{TPLT}{#1}} %For explanation of the template
\newcommand{\an}[1]{\authornote{cyan}{Author}{#1}}

%% Common Parts

\newcommand{\progname}{ProgName} % PUT YOUR PROGRAM NAME HERE
\newcommand{\authname}{Team \#, Team Name
\\ Student 1 name
\\ Student 2 name
\\ Student 3 name
\\ Student 4 name} % AUTHOR NAMES                  

\usepackage{hyperref}
    \hypersetup{colorlinks=true, linkcolor=blue, citecolor=blue, filecolor=blue,
                urlcolor=blue, unicode=false}
    \urlstyle{same}
                                


\begin{document}

\begin{table}[hp]
\caption{Revision History} \label{TblRevisionHistory}
\begin{tabularx}{\textwidth}{llX}
\toprule
\textbf{Date} & \textbf{Developer(s)} & \textbf{Change}\\
\midrule
11/20/2022 & Anton Kanugalawattage & Revised PoC Demo Plan\\
  & Dipendra Subedi & \\
  & Youssef Rizkalla &\\
  & Tamas Leung & \\
  & Zhiming Zhao & \\
\midrule
09/20/2022 & Anton Kanugalawattage & Initial Document\\
  & Dipendra Subedi & \\
  & Youssef Rizkalla &\\
  & Tamas Leung & \\
  & Zhiming Zhao & \\
\bottomrule
\end{tabularx}
\end{table}

\newpage

\maketitle

% \wss{Put your introductory blurb here.}

\section{Team Meeting Plan}
\begin{itemize}
\item Weekly meetings on Tuesday 6:30pm
\item Adhoc calls when necessary
\item Every member will add items to the meeting agenda
\end{itemize}

\section{Team Communication Plan}
\begin{itemize}
\item Discord
\item Github Issues
\item Github Project Board
\end{itemize}

\section{Team Member Roles}
\begin{itemize}
\item Scrum Master: Tamas Leung
\begin{itemize}
    \item Will focus on making sure people are doing their tasks
    \item Focus on re-prioritizing features and bugs
\end{itemize}
\item Front-end Lead: Anton Kanugalawattage
\begin{itemize}
    \item Designs architecture for front-end of the system
    \item Oversees and reviews frontend pull requests
    \item Works on Web Socket connections
\end{itemize}
\item Back-end Lead: Youssef Rizkalla
\begin{itemize}
    \item Designs architecture for back-end of the system
    \item Oversees and reviews back-end pull requests
\end{itemize}
\item Design Lead: Dipendra Subedi
\begin{itemize}
    \item Designing backend APIs
    \item Designing database schemas
    \item Ensures HCI requirements are met
\end{itemize}
\item Testing Lead: Zhiming Zhao
\begin{itemize}
    \item Create test plans during different development phases
    \item Run tests on various components and features in order to identify and fix issues
    \item Analyzing results and identifying the cause of problems
\end{itemize}
\end{itemize}   
All team members will participate in creating features all around the tech stack. The roles are intended for each member to have a focus-area, which can later change depending on the team's needs.

\section{Workflow Plan}

\begin{itemize}
    \item Git: \begin{itemize}
        \item All features are developed onto separate branches
        \item Each feature branch can only be merged into the master branch via a pull request
        \item Each pull request must be at least approved by at least two members
        \item Github Actions CI to auto test new pull requests
    \end{itemize}
	\item Issues: \begin{itemize}
	    \item Tasks will be placed into issues board based on priority, type and technology
        \item Issue Types:
            \begin{itemize}
                \item Feature
                \item Bug
                \item DevOps
            \end{itemize}
        \item Technology: 
        \begin{itemize}
                \item Front End
                \item Back End
                \item Database
            \end{itemize}
        \item Priority:
        \begin{itemize}
            \item P0: Needs to be addressed this week
            \item P1: Needs to be addressed this month
            \item P2: Will be addressed whenever there is no higher priority items available
        \end{itemize}
        \item Issues will be prioritized on a bi-weekly basis during scrum meetings
        \item Use Github Project board, to rearrange issues and assign task
	\end{itemize}
\end{itemize}

\section{Proof of Concept Demonstration Plan}

% What is the main risk, or risks, for the success of your project?  What will you
% demonstrate during your proof of concept demonstration to convince yourself that
% you will be able to overcome this risk?

% old
% Our main risk is not knowing if our back-end server will be able to handle multiple concurrent compilations at the same time.

% We will demonstrate running a multiplayer game with concurrent compilation of programs, as the multiplayer aspect is the core of the project.

% new

% Risk?

Our main risk is not knowing if our back-end server can execute given code and evaluate it for correction for a set problem.

We will demonstrate that a front-end can communicate with a back-end server to compile code, compute and return results based on evaluation against set test cases. The front-end will show a code editor with syntax highlighting, a basic homepage, as well as problem descriptions. The front-end will also call the back end for a list of problems against a problems API to demonstrate that we can effectively store and retrieve problems.

The demonstration will also show a documentation page to show diligence in knowledge transfer.

\section{Technology}

\begin{itemize}
\item Languages: JavaScript/TypeScript
\item Frontend Framework: Angular
\item Backend Framework: Node.js / Express
\item Testing:
   \begin{itemize}
       \item Backend: Jest
       \item Frontend: Jasmine
       \item End-to-End: TestCafe
   \end{itemize}
\item Code Coverage: Included with test frameworks
\item Github Actions CI 
    \begin{itemize}
        \item Auto build and test new pull requests
        \item Auto Linting
        \item Auto Formatting
    \end{itemize}
\item Performance Testing: 
    \begin{itemize}
       \item Backend: Postman Performance Testing
       \item Frontend: Chrome Dev Tools Performance Tester
   \end{itemize}
\item Database: NoSQL Database (MongoDB)
\item Deployment: Amazon Web Services (AWS)
\item Libraries / Tools
\begin{itemize}
    \item Web Sockets: for real time connections
    \item Auth0: for authentication \& identity management
    \item HTTP: Endpoints for CRUD operations
\end{itemize}
\end{itemize}

\section{Coding Standard}

\begin{itemize}
    \item Linter: ESlint
    \item Code formatter: Prettier
    \item The \href{https://github.com/airbnb/javascript}{Airbnb style guide} will be enforced during code review for all back-end code
    \item The \href{https://angular.io/guide/styleguide}{official style guide} will be enforced during code review for all front-end code
\end{itemize}

\section{Project Scheduling}

Major milestones will be placed onto deadlines. Each task will be assigned bi-weekly at the scrum meetings.

\begin{itemize} 
    \item CI/CD Setup: October 10st
    \item Backend Architecture: October 16st
    \item Frontend Development for PoC: October 21th
    \item WebSocket Connection: October 26th
    \item PoC Demo Completion: November 7th
\end{itemize}

\end{document}